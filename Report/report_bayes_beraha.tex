\documentclass[12pt,a4paper]{article}


\usepackage[utf8]{inputenc}
\usepackage[latin1]{inputenc}%
\usepackage[a4paper,left=3cm,right=2cm,top=2.5cm,bottom=2.5cm]{geometry}
\usepackage{amsthm,amssymb}
\usepackage{graphicx}
\usepackage[fleqn]{amsmath}
\usepackage{caption}
\usepackage{subcaption} 
\usepackage{subfig}
\usepackage{mwe}
\usepackage{bm}
\setlength{\parindent}{4em}
\setlength{\parskip}{1em}
\renewcommand{\baselinestretch}{1.3}

\newenvironment{theorem}[2][Theorem]{\begin{trivlist}
		\item[\hskip \labelsep {\bfseries #1}\hskip \labelsep {\bfseries #2.}]}{\end{trivlist}}
\newenvironment{lemma}[2][Lemma]{\begin{trivlist}
		\item[\hskip \labelsep {\bfseries #1}\hskip \labelsep {\bfseries #2.}]}{\end{trivlist}}
\newenvironment{exercise}[2][Exercise]{\begin{trivlist}
		\item[\hskip \labelsep {\bfseries #1}\hskip \labelsep {\bfseries #2.}]}{\end{trivlist}}
\newenvironment{reflection}[2][Reflection]{\begin{trivlist}
		\item[\hskip \labelsep {\bfseries #1}\hskip \labelsep {\bfseries #2.}]}{\end{trivlist}}
\newenvironment{proposition}[2][Proposition]{\begin{trivlist}
		\item[\hskip \labelsep {\bfseries #1}\hskip \labelsep {\bfseries #2.}]}{\end{trivlist}}
\newenvironment{corollary}[2][Corollary]{\begin{trivlist}
		\item[\hskip \labelsep {\bfseries #1}\hskip \labelsep {\bfseries #2.}]}{\end{trivlist}}

\newenvironment{definition}[2][Definition]{\begin{trivlist}
		\item[\hskip \labelsep {\bfseries #1}\hskip \labelsep {\bfseries #2.}]}{\end{trivlist}}

\newcommand{\floor}[1]{\lfloor #1 \rfloor}


\begin{document}
\linespread{2}
\title{Bayesian inference for differential equations}%replace X with the appropriate number
\author{Mario Beraha\\ %replace with your name
	Bayesian Statistics   prof. A. Guglielmi} 

\maketitle

\begin{center}
	\includegraphics[width=.5\linewidth]{../grafici/logo_polimi.jpg}
\end{center}

\newpage

\section{Introduzione}
L'utilizzo di equazioni differenziali per modellizzare sistemi chimici e biologici ha, al giorno d'oggi, una storia ben consolidata; tuttavia, solo recentemente si è iniziato a porsi, e conseguentemente risolvere, il problema legato all'incertezza sul modello.

In molte applicazioni, la modellizzazione del sistema in considerazione non è ben caratterizzata nè dal punto di vista della scelta del modello "corretto" nè per quanto riguarda i valori dei parametri dello stesso.

L'incertezza in entrambi questi aspetti deve essere presa in considerazione in maniera sistematica se si vogliono fare simulazioni, o predizioni, in modo da evitare di giungere a conclusioni non giustificate o ingiustificatamente troppo ottimiste data l'incertezza di base.

Come mostrato nelle prossime pagine, la metodologia dell'inferenza bayesiana fornisce un setting coerente per quantificare l'incertezza propagatasi nel modello.

\section{Inferenza bayesiana}

Si considerano i dati osservati $ \mathcal{D} \{\bm{y}, \bm t \} \ \bm y \ \in \ \mathbb{R}^N, \ t \ \in \ \mathbb{R}$, dove $\bm y$ è l'osservazione legata a un certo stimolo $t$, che può essere visto semplicemente come il tempo. L'obiettivo è di stabilire una relazione funzionale $\phi: \ t \mapsto \bm y$.

Per fare ciò, definiamo la classe dei modelli che verranno presi in considerazione $\textit M = \{M_1,M_2 \dots M_k \}$ a cui è assegnata una prior (discreta) $\pi = \{\pi (M_j) \}$

Se l'obiettivo è quello di fare previsione su un possibile valore $y_*$ dato uno specifico $t_*$, ossia calcolare $p(y_* | t_*, \mathcal{D})$, sia l'incertezza sui parametri che quella sul modello deve essere presa in considerazione.

\'E facile osservare che:
\begin{equation}
	p(y_* | t_*, \mathcal{D}) = \sum_{k \in M} p(y_* | t_*, \mathcal{D}, M_k) \pi(M_k|\mathcal{D})	
\end{equation}
Ossia che la distribuzione predittiva è esprimibile come la media delle predittive di ogni modello. 
Usando il teorema di Bayes per modelli dominati, si ha l'espressione della posterior
\begin{equation}
	\pi(M_k|\mathcal{D}) = \frac{p(\mathcal{D}|M_k) \pi(M_k)}{\sum_{ \in {M}} p(\mathcal{D}|M_l) \pi(M_l)}	
\end{equation}


\end{document}